\documentclass[journal]{IEEEtran}
\usepackage{blindtext}
\usepackage{graphicx}
\hyphenation{op-tical net-works semi-conduc-tor}
\begin{document}
\title{Neurodegenerative Diseases Study Guide}
\author{Rakesh~Chatrath}
		% John~Doe,~\IEEEmembership{Feloow,~OSA}}
% \thanks{M. Shell is with the Department
% of Electrical and Computer Engineering, Gerogia Institute of Technology, Atlanta,
% GA, 30332 USA e-mail: (see http://michaelshell.org/contact.html).
% }
% \thanks{J. Doe and J. Doe are with Anonymous University.}
% \thanks{Manuscript received April 19. 2005; revised January 11, 2007.
% }}
% \markboth{Journal of \LaTeX\ Class Files,~Vol.~6, No.~1, Januray~2007
% }
% {Shell \MakeLowercase{\textit{et. al}}: Bare Demo of IEEEtran.cls for Journals}
\maketitle
% \begin{abstract}
% \blindtext[1]
% \end{abstract}
% \begin{IEEEkeywords}
% IEEEtran, journal, \LaTex, paper, template.
% \end{IEEEkeywords}
% \IEEEpeerreviewmaketitle
\section{Bipolar Disorder}
\begin{abstract}
Disease in which the patient has two diferent mental states: mania and depressive. Patient fluctuates in between the two. Diagnosis is difficult and the cause is not known, but could be related to inherited genetic traits, physiological abnormalities in the brain, chemical imbalance with neurological transmitters abd initiated by environmental stressors.
\end{abstract}
\begin{IEEEkeywords}
Mania, Depressive
\end{IEEEkeywords}
\subsection{Causes}
The direct cause of Bipolar Disorder is unknown, however, it can be related to inherited genetic traits, physiological differences in the brain, chemical imbalance with neurotransmitters and initiated by the environmental stressors.
\\
\vspace{2 mm}
Genetic causes:
\begin{itemize}
\item Familial: Half the people with bipolar disorder have a family member with it
\item A person with one parent with it has a 15-25 percent chance of developing it
\item A person with a fraternal twin with it has a 25 percent chance of developing, same risks as if both parents have it
\item A person with an identical twin with bipolar disorder has eightfold chance
\end{itemize}
\vspace{2 mm}
Neurochemcial Causes:
\begin{itemize}
\item Dysfunction with neurotransmitters and their respective chemicals
\item Norepinephrine
\item Serotonin
\item May lie dormant and can be activated on its own or triggered by external factors
\end{itemize}
Environmental Factors:
\begin{itemize}
\item Life event extremities
\item Altered health habits such as alcohol or drug abuse
\item Underdiagnosis in the past could explain the trend of BD at earlier ages
\item Substance abuse may not be a cause, but in can worsen the depressive state
\end{itemize}
\begin{IEEEkeywords}
Genetical/Familial, Neurochemical, Environmental
\end{IEEEkeywords}
\subsection{Symptoms}
Bipolar disorder is generally characterized by two episodes:
\\
\begin{itemize}
\item Manic: Extreme Happiness/Giddiness
\item Depressive: Extreme Sadness (Depression)
\end{itemize}
\vspace{2 mm}
The patient will fluctuate between these two episodes in violent mood swings. They can be in one mood for the whole day than switch to the other one spontaneously. It is dependent on the situation and the individual rather than anything concrete or genetically related.
\begin{IEEEkeywords}
Manic, Depressive
\end{IEEEkeywords}
\subsection{Treatments}


\section{Disease}
\begin{abstract}
\blindtext[1]
\end{abstract}
\begin{IEEEkeywords}
\blindtext[1]
\end{IEEEkeywords}
\blindtext[1]
\subsection{Subsection of Disease}
\begin{abstract}
\blindtext[1]
\end{abstract}
\begin{IEEEkeywords}
\blindtext[1]
\end{IEEEkeywords}
\blindtext
% \appendices
% \section{Proof of the First Zonklar Equation}
% Some text for the appendix
% \section*{Acknowledgement}


% The authors would like to thank...
% \ifCLASSOPTIONcaptionsoff
% 	\newpage
% \fi
% \begin{thebibliography}{1}
% \bibitem{IEEEhowto:kopka}
% H.~Kopka and P.~W. Daly, \emph{A Guide to \LaTeX}, 3rd~ed.\hskip 1em plus
%   0.5em minus 0.4em\relax Harlow, England: Addison-Wesley, 1999.
% \end{thebibliography}
% \begin{IEEEbiography}[{\includegraphics[width=1in,height=1.25in,clip,keepaspectratio]{picture}}]{John Doe}
% \blindtext
% \end{IEEEbiography}
\end{document}