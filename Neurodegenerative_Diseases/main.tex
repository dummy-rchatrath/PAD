\documentclass[journal, 12pt]{IEEEtran}
\usepackage{blindtext}
\usepackage{graphicx}
\usepackage{assymb}
\usepackage{enumitem}
\hyphenation{op-tical net-works semi-conduc-tor}
\begin{document}
\title{Neurodegenerative Diseases Study Guide}
\author{Rakesh~Chatrath}
		% John~Doe,~\IEEEmembership{Feloow,~OSA}}
% \thanks{M. Shell is with the Department
% of Electrical and Computer Engineering, Gerogia Institute of Technology, Atlanta,
% GA, 30332 USA e-mail: (see http://michaelshell.org/contact.html).
% }
% \thanks{J. Doe and J. Doe are with Anonymous University.}
% \thanks{Manuscript received April 19. 2005; revised January 11, 2007.
% }}
% \markboth{Journal of \LaTeX\ Class Files,~Vol.~6, No.~1, Januray~2007
% }
% {Shell \MakeLowercase{\textit{et. al}}: Bare Demo of IEEEtran.cls for Journals}
\maketitle
% \begin{abstract}
% \blindtext[1]
% \end{abstract}
% \begin{IEEEkeywords}
% IEEEtran, journal, \LaTex, paper, template.
% \end{IEEEkeywords}
% \IEEEpeerreviewmaketitle
\section{Autism and Asperger's}
\begin{abstract}
Range of neurodevelopmental disorders called Autism Spectrum Disorders (ASD). Asperger's is on the lower end of the spectrum while Autism can range from milder to severe. These diseases can also be characterized under Pervasive Developmental Disorder Not Otherwise Specified (PDD-NOS). Autism is characterized by social, behavioral and communicational impairment. Asperger's also implies social and behavioral impairments but communication is fine. 
\end{abstract}
\begin{IEEEkeywords}
Neurdevelopmental Disorders, Autism Spectrum Disorders (ASD), Prevasive Developmental Disorder Not Otherwise Specified (PDD-NOS), Autism, Asperger's 
\end{IEEEkeywords} 
\subsection{Symptoms and Characterizations}
In general, Autism is characterized by social, behavioral and communicational impediments. Asperger's is characterized by social and behavioral impediments but NOT speech impairment. 
\begin{itemize}
\item Autism: Social, Behavioral, Speech
\item Asperger's: Similar to Autism but without communication. Also has Obsessiveness. 
\end{itemize}
\noindent\parbox[t]{2.1in}{\raggedright%
\textbf{\textit{Autism Syndrome}}
\begin{enumerate}[topsep=0pt,itemsep=-2pt,leftmargin=7pt]
\item Communication difficulties
\item Avoid eye contact during conversation
\item Desire to be alone 
\item Trouble with feelings 
\item Tendency to echo words or phrases
\item Speech delay 
\item Witdrawal from socializing
\end{enumerate}
}% 
\parbox[t]{2.1in}{\raggedright%
\textbf{\textit{Asperger's Syndrome}}
\begin{enumerate}[topsep=0pt,itemsep=-2pt,leftmargin=7pt]
\item Impaired nonverbal skills
\item Trouble maintaining relationships
\item May have a lack of 
\\
empathy
\item May be physically 
\\
awkward
\item Above average memory
\item Struggle with abstract concepts 
\item Intelligence normal to 
\\
above average 
\item Motor skill delays
\item Want to interact with 
\\
society but doesn't know how 
\end{enumerate}
}
\parbox[t]{2.4in}{\raggedright%
\textbf{\textit{Combined Symptoms}}
\begin{enumerate}[topset=0pt,itemsep=-2pt,leftmargin=13pt]
\item Sensory disorders
\item Social impairments
\item Fixated, obsessive interests
\item Repetitive behavior
\item Adherence to routine
\end{enumerate}
}%
\\
\vspace{2 mm}
\subsubsection{Co-Ocurring Conditions}
\begin{itemize} 
\item Anxiety 
\item Depression 
\item OCD
\item Tourette's
\item ADD/ADHD
\item Fragile X Syndrome
\item Tuberous Sclerosis
\end{itemize}
\subsubsection{Diagnosis}
\begin{itemize}
\item Start with children
\begin{itemize}
\item Lack of responsiveness
\end{itemize}
\item Questionnaire and screening administered
\item Comprehensive evaluation witha  multidisplinary team 
\item Cognitive and language testing 
\end{itemize}
\begin{IEEEkeywords}
Speech impediment: Autism, Obsessiveness no speech impediment: Asperger's 
\end{IEEEkeywords}
\subsection{Treatment}
Autism and Asperger's being psychological disorders, do not really have a direct treatment. Rather, their is therapy to help absolve some of the symptoms, and medication for some of the Co-Occurring Conditions. 
\begin{itemize}
\item Therapy and behavioral interventions
\item Medication for treatment of co-occurring conditions
\begin{itemize}
\item Medicotian for ADD and OCD
\end{itemize}
\item Antipsychotic medication for severe behavior issues
\item Preemptive care improves development
\end{itemize}
\begin{IEEEkeywords}
Therapy, Medication for ADD/OCD, Preemptive care
\end{IEEEkeywords}
\subsection{Causes}
\subsubsection{Risk Factors}
\begin{itemize}
\item Occurs 5 times more in males
\item Genetic predisposition and environmental factors: (TF does this even mean)
\item Immune dysfunction and neurological abnormalities
\item Genetic and chromosomal conditions 
\item Child of an older couple == higher risk 
\item Cause unknown
\end{itemize}
\subsubsection{Genetic Theories}
\begin{itemize}
\item Immune dysfunction and neurological abnormalities
\item X linked, common in males
\item Genetical cause more significant in Asperger's than Autism
\item Autism may result from a combination of genetics and brain injury 
\end{itemize}
\subsubsection{Neurological Theories}
Main neurological theories is related to abnormalities in teh limbic system. That is, the cerebellum, hippocampus, and amygdala. The limbic system is responsible for emotional and behavioral developemnt. 
\begin{itemize}
\item Dysfunction in serotonergic system: 
\begin{itemize} 
\item Higher levels of serotonin
\end{itemize}
\item Neural Overconnectivity
\item Cerebral Overgrowth
\item Circuitry abnormalities in cerebellum, hippocampus, and limbic regions
\item Loss of Purkinje cells in hippocampus, amygdala, and cerebellum
\item Abnormal assembly of dendritic spines 
\begin{itemize}
\item Long and thin spines yielding: 
\end{itemize}
\item Altered calcium signaling 
\item Mirror neuron dysfunction 
\begin{itemize}
\item Neurons responsible for evolution of language, empothy and conversational skills
\end{itemize}
\end{itemize}
\subsubsection{Immunological Theories}
\begin{itemize}
\item Decreased levels of apoptosis 
\item Metabolic Defects
\item Autoimmune diseases
\item Viral infections early childhood or prenatal development
\item Excessive or improper vaccination OF an immunocompromised child
\item Leaky gut
\item Gut dysbiosis
\end{itemize}
\subsubsection{Prenatal Theories}
\begin{itemize}
\item Early birth
\item Exposure to pathogens prenatally
\item Heavy metal toxicity
\end{itemize}
\begin{IEEEkeywords}
X linked, Higher levels of serotonin, Limbic system, Mirror neuron
\end{IEEEkeywords}
\subsection{Connections to Other Diseases}
No connections really to other diseases. 
\section{Left Neglect Disorder}
\begin{abstract}
 
\section{Bipolar Disorder}
\begin{abstract}
Disease in which the patient has two diferent mental states: mania and depressive. Patient fluctuates in between the two. Diagnosis is difficult and the cause is not known, but could be related to inherited genetic traits, physiological abnormalities in the brain, chemical imbalance with neurological transmitters abd initiated by environmental stressors.
\end{abstract}
\begin{IEEEkeywords}
Mania, Depressive
\end{IEEEkeywords}
\subsection{Causes}
The direct cause of Bipolar Disorder is unknown, however, it can be related to inherited genetic traits, physiological differences in the brain, chemical imbalance with neurotransmitters and initiated by the environmental stressors.
\\
\vspace{2 mm}
Genetic causes:
\begin{itemize}
\item Familial: Half the people with bipolar disorder have a family member with it
\item A person with one parent with it has a 15-25 percent chance of developing it
\item A person with a fraternal twin with it has a 25 percent chance of developing, same risks as if both parents have it
\item A person with an identical twin with bipolar disorder has eightfold chance
\end{itemize}
\vspace{2 mm}
Neurochemcial Causes:
\begin{itemize}
\item Dysfunction with neurotransmitters and their respective chemicals
\item Norepinephrine
\item Serotonin
\item May lie dormant and can be activated on its own or triggered by external factors
\end{itemize}
Environmental Factors:
\begin{itemize}
\item Life event extremities
\item Altered health habits such as alcohol or drug abuse
\item Underdiagnosis in the past could explain the trend of BD at earlier ages
\item Substance abuse may not be a cause, but in can worsen the depressive state
\end{itemize}
\begin{IEEEkeywords}
Genetical/Familial, Neurochemical, Environmental
\end{IEEEkeywords}
\subsection{Symptoms}
Bipolar disorder is generally characterized by two episodes:
\\
\begin{itemize}
\item Manic: Extreme Happiness/Giddiness
\begin{itemize}
\item Hyperactive Mannerisms
\end{itemize}
\item Depressive: Extreme Sadness (Depression)
\begin{itemize}
\item Slower mannerisms
\end{itemize}
\end{itemize}
\vspace{2 mm}
The patient will fluctuate between these two episodes in violent mood swings. They can be in one mood for the whole day than switch to the other one spontaneously. It is dependent on the situation and the individual rather than anything concrete or genetically related.
\begin{IEEEkeywords}
Manic, Depressive
\end{IEEEkeywords}
\subsection{Treatments}
Treatment options include: 
\begin{itemize}
\item Medications
\begin{itemize}
\item Mood Stabilizer (Lithium)
\item Antispychotics (Olanzapine)
\item Antidepressants (Fluoxetine)
\item Antidepressant-antipsychotics
\item Anti-anxieties
\end{itemize}
\item Psychotherapy: General term for treating mental health problems by talking with a psychiatrist  
\item Electroconvulsive Therapy (ECT): Electrical currents passed through the brain to intentially cause a seizure. This wil alter the brain chemistry and thereby help revert certain mental illness symptoms. 
\end{itemize}
\begin{IEEEkeywords}
Mood Stabilizers, Antipsychotics, Andtidepressants, Antidepressants-antipsychotics, anti-anxieties, psychotherapy, Electroconvulsive Therapy (ECT)
\end{IEEEkeywords}
\subsection{Special Characteristics}
Chracterized by drastic mood swings from periods of extreme highs to extreme lows. This complicates treatment options. The nature of the disease makes diagnosis difficult: 
\begin{itemize}
\item Similar symptoms to other conditions
\item Difficulty with dealing with patients
\end{itemize}
\begin{IEEEkeywords}
Mood Swings
\end{IEEEkeywords}
\subsection{Relation to Other Diseases}
\begin{itemize}
\item Autism and Asperger
\begin{itemize}
\item Similar symptoms resulting in misdiagnosis
\end{itemize}
\item Depression and Multiple Personality Disorder
\begin{itemize}
\item Symptoms can coincide or be related
\end{itemize}
\item Effect of prescription and non prescription drugs an the brain
\begin{itemize}
\item Can cause or enhance similar symptoms
\end{itemize}
\item Left Neglect
\begin{itemize}
\item Both neurological, but very little similarity
\end{itemize}
\end{itemize}
\subsection{Pathology}
Has genetic linkage. Regions of interest include mutations on the chromosomes: 
\begin{itemize}
\item 4p16
\item 12q23-q24
\item 16p13
\item 21q22
\item Xq24-q26
\end{itemize}
\begin{IEEEkeywords}
Chromosomal Mutation
\end{IEEEkeywords}
\subsection{Pathophysiology}
There is imbalance in neurotransmitters in the brain that lead to mood alterations. Furthermore, the prefrontal cortex (responsible for problem solving and decision making) in adults tends to be smaller and function less. 
\begin{itemize}
\item Area matures during adolescence, which would explain appearance of disorder around a person's teen years
\end{itemize}
\begin{IEEEkeywords}
Mood Disorder, prefrontal cortex, neurotransmitters 
\end{IEEEkeywords}

\section{}
\blindtext[1]
% \appendices
% \section{Proof of the First Zonklar Equation}
% Some text for the appendix
% \section*{Acknowledgement}


% The authors would like to thank...
% \ifCLASSOPTIONcaptionsoff
% 	\newpage
% \fi
% \begin{thebibliography}{1}
% \bibitem{IEEEhowto:kopka}
% H.~Kopka and P.~W. Daly, \emph{A Guide to \LaTeX}, 3rd~ed.\hskip 1em plus
%   0.5em minus 0.4em\relax Harlow, England: Addison-Wesley, 1999.
% \end{thebibliography}
% \begin{IEEEbiography}[{\includegraphics[width=1in,height=1.25in,clip,keepaspectratio]{picture}}]{John Doe}
% \blindtext
% \end{IEEEbiography}
\end{document}