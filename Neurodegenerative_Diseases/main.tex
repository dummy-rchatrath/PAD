\documentclass[journal]{IEEEtran}
\usepackage{blindtext}
\usepackage{graphicx}
\usepackage{assymb}
\usepackage{enumitem}
\hyphenation{op-tical net-works semi-conduc-tor}
\begin{document}
\title{Neurodegenerative Diseases Study Guide}
\author{Rakesh~Chatrath}
		% John~Doe,~\IEEEmembership{Feloow,~OSA}}
% \thanks{M. Shell is with the Department
% of Electrical and Computer Engineering, Gerogia Institute of Technology, Atlanta,
% GA, 30332 USA e-mail: (see http://michaelshell.org/contact.html).
% }
% \thanks{J. Doe and J. Doe are with Anonymous University.}
% \thanks{Manuscript received April 19. 2005; revised January 11, 2007.
% }}
% \markboth{Journal of \LaTeX\ Class Files,~Vol.~6, No.~1, Januray~2007
% }
% {Shell \MakeLowercase{\textit{et. al}}: Bare Demo of IEEEtran.cls for Journals}
\maketitle
% \begin{abstract}
% \blindtext[1]
% \end{abstract}
% \begin{IEEEkeywords}
% IEEEtran, journal, \LaTex, paper, template.
% \end{IEEEkeywords}
% \IEEEpeerreviewmaketitle
\section{Autism and Asperger's}
\begin{abstract}
Range of neurodevelopmental disorders called Autism Spectrum Disorders (ASD). Asperger's is on the lower end of the spectrum while Autism can range from milder to severe. These diseases can also be characterized under Pervasive Developmental Disorder Not Otherwise Specified (PDD-NOS). Autism is characterized by social, behavioral and communicational impairment. Asperger's also implies social and behavioral impairments but communication is fine.
\end{abstract}
\begin{IEEEkeywords}
Neurdevelopmental Disorders, Autism Spectrum Disorders (ASD), Prevasive Developmental Disorder Not Otherwise Specified (PDD-NOS), Autism, Asperger's
\end{IEEEkeywords}
\subsection{Symptoms and Characterizations}
In general, Autism is characterized by social, behavioral and communicational impediments. Asperger's is characterized by social and behavioral impediments but NOT speech impairment.
\begin{itemize}
\item Autism: Social, Behavioral, Speech
\item Asperger's: Similar to Autism but without communication. Also has Obsessiveness.
\end{itemize}
\noindent\parbox[t]{2.1in}{\raggedright%
\textbf{\textit{Autism Syndrome}}
\begin{enumerate}[topsep=0pt,itemsep=-2pt,leftmargin=7pt]
\item Communication difficulties
\item Avoid eye contact during conversation
\item Desire to be alone
\item Trouble with feelings
\item Tendency to echo words or phrases
\item Speech delay
\item Witdrawal from socializing
\end{enumerate}
}%
\parbox[t]{2.1in}{\raggedright%
\textbf{\textit{Asperger's Syndrome}}
\begin{enumerate}[topsep=0pt,itemsep=-2pt,leftmargin=7pt]
\item Impaired nonverbal skills
\item Trouble maintaining relationships
\item May have a lack of
\\
empathy
\item May be physically
\\
awkward
\item Above average memory
\item Struggle with abstract concepts
\item Intelligence normal to
\\
above average
\item Motor skill delays
\item Want to interact with
\\
society but doesn't know how
\end{enumerate}
}
\parbox[t]{2.4in}{\raggedright%
\textbf{\textit{Combined Symptoms}}
\begin{enumerate}[topset=0pt,itemsep=-2pt,leftmargin=13pt]
\item Sensory disorders
\item Social impairments
\item Fixated, obsessive interests
\item Repetitive behavior
\item Adherence to routine
\end{enumerate}
}%
\\
\vspace{2 mm}
\subsubsection{Co-Ocurring Conditions}
\begin{itemize}
\item Anxiety
\item Depression
\item OCD
\item Tourette's
\item ADD/ADHD
\item Fragile X Syndrome
\item Tuberous Sclerosis
\end{itemize}
\subsubsection{Diagnosis}
\begin{itemize}
\item Start with children
\begin{itemize}
\item Lack of responsiveness
\end{itemize}
\item Questionnaire and screening administered
\item Comprehensive evaluation witha  multidisplinary team
\item Cognitive and language testing
\end{itemize}
\begin{IEEEkeywords}
Speech impediment: Autism, Obsessiveness no speech impediment: Asperger's
\end{IEEEkeywords}
\subsection{Treatment}
Autism and Asperger's being psychological disorders, do not really have a direct treatment. Rather, their is therapy to help absolve some of the symptoms, and medication for some of the Co-Occurring Conditions.
\begin{itemize}
\item Therapy and behavioral interventions
\item Medication for treatment of co-occurring conditions
\begin{itemize}
\item Medicotian for ADD and OCD
\end{itemize}
\item Antipsychotic medication for severe behavior issues
\item Preemptive care improves development
\end{itemize}
\begin{IEEEkeywords}
Therapy, Medication for ADD/OCD, Preemptive care
\end{IEEEkeywords}
\subsection{Causes}
\subsubsection{Risk Factors}
\begin{itemize}
\item Occurs 5 times more in males
\item Genetic predisposition and environmental factors: (TF does this even mean)
\item Immune dysfunction and neurological abnormalities
\item Genetic and chromosomal conditions
\item Child of an older couple == higher risk
\item Cause unknown
\end{itemize}
\subsubsection{Genetic Theories}
\begin{itemize}
\item Immune dysfunction and neurological abnormalities
\item X linked, common in males
\item Genetical cause more significant in Asperger's than Autism
\item Autism may result from a combination of genetics and brain injury
\end{itemize}
\subsubsection{Neurological Theories}
Main neurological theories is related to abnormalities in teh limbic system. That is, the cerebellum, hippocampus, and amygdala. The limbic system is responsible for emotional and behavioral developemnt.
\begin{itemize}
\item Dysfunction in serotonergic system:
\begin{itemize}
\item Higher levels of serotonin
\end{itemize}
\item Neural Overconnectivity
\item Cerebral Overgrowth
\item Circuitry abnormalities in cerebellum, hippocampus, and limbic regions
\item Loss of Purkinje cells in hippocampus, amygdala, and cerebellum
\item Abnormal assembly of dendritic spines
\begin{itemize}
\item Long and thin spines yielding:
\end{itemize}
\item Altered calcium signaling
\item Mirror neuron dysfunction
\begin{itemize}
\item Neurons responsible for evolution of language, empothy and conversational skills
\end{itemize}
\end{itemize}
\subsubsection{Immunological Theories}
\begin{itemize}
\item Decreased levels of apoptosis
\item Metabolic Defects
\item Autoimmune diseases
\item Viral infections early childhood or prenatal development
\item Excessive or improper vaccination OF an immunocompromised child
\item Leaky gut
\item Gut dysbiosis
\end{itemize}
\subsubsection{Prenatal Theories}
\begin{itemize}
\item Early birth
\item Exposure to pathogens prenatally
\item Heavy metal toxicity
\end{itemize}
\begin{IEEEkeywords}
X linked, Higher levels of serotonin, Limbic system, Mirror neuron
\end{IEEEkeywords}
\subsection{Connections to Other Diseases}
No connections really to other diseases.
\section{Left Neglect Disorder}
\begin{abstract}
Left neglect, hemispatial neglect or Right Hemisphere Brain Damage is a perceptual disorder in which the person ingores or has difficulting perceiving anything on the left side. For example, they could not be able to use their left arm and leg as much, eat only food on the right side, or read words only on the right side of the page.
\end{abstract}
\begin{IEEEkeywords}
Left Neglect, Hemispatial neglect, Right Hemisphere Brain Damage, Perceptual Disorder
\end{IEEEkeywords}
\subsection{Background Information}
The brain has two hemispheres, the right and left.
\begin{itemize}
\item Left Hemisphere: Responsible for language function
\item Right Hemisphere: Responsible for numerous actions including:
\begin{itemize}
\item Memory
\item Attention
\item Reasoning
\end{itemize}
\end{itemize}
Damage to the Right Hemisphere will cause a disruption in most cognitive skills. Because of this, the person will not realize that they are even experiencing any of the problems that will arise or that they have brain damage.
\begin{IEEEkeywords}
Right Hemisphere, Left Hemisphere
\end{IEEEkeywords}
\subsection{Physiology}
\subsubsection{Temporal-Parietal Junction}
A cause of Left Neglect can be attributed to a stroke. A stroke occurs when there is an infarction of the brain, causing those brain cells to die and that part of the brain to be damaged. Sometimes, the part of the brain that is afflicted by the stroke can be the temporal-parietal junction and posterior parietal cortex. The temporal-parietal junction of the brain is responsible for self-awareness of the person.
\begin{itemize}
\item Orient body in space
\item Coordination of the body as the person wishes
\end{itemize}
Furthermore, the temporo junction can be responsible for emotional processing and moral judgments.
\begin{IEEEkeywords}
Temporal-Parietal Junction, Posterior parietal cortex,
\end{IEEEkeywords}
\subsubsection{Posterior Parietal Cortex}
Similarly, Left Neglect occurs when a person experience damage to the Posterior Parietal Cortex. The post parietal cortex is used for voluntary movement. It responds to a stimulus and uses visual displays and other factors to determine its position of the body and target in space. There are two parts of the posterior parietal coretex: area 5 and area 7. Area 5 responds tp any sensory stimuli while area 7 responds to any visusal stimuli. These two work in unison to form the posterior parietal cortex and perform any voluntary movements.
\begin{IEEEkeywords}
Voluntary moevement, Area 5, Area 7
\end{IEEEkeywords}
\subsection{Causes}
There is a high probability of developing Left Neglect after a stroke. Damage as a result of the stroke to the temporo-parietal junction and the posterior parietal cortex cause Left Neglect. If these two structures are damaged, then the person loses coordination and voluntary movement. The combination of these two results in Left Neglect: the disorder in which the person ignores or does not process movement and objects in the left side.
\begin{IEEEkeywords}
Damage to the right hemisphere includes: stroke and traumatic incident
\end{IEEEkeywords}
\subsection{Symptoms}
\begin{itemize}
\item Paying less attention to their left side
\item Ignoring their left arm and left leg
\item Only eating food from the right side of their plate
\item Reading only the right side of a page and often losing their place
\item Emotional distress
\end{itemize}
\subsection{Treatments}
There is no clinical way to treat someone with Left Neglect. The best treatment is to offer support and slowly guide them to using their left side. Therapies are also effective with this, such as covering the right eye forcing the person to use the left eye.
\begin{IEEEkeywords}
Therapy
\end{IEEEkeywords}
\subsection{Relationship to Other Diseases}
\begin{itemize}
\item Prescription Drugs
\begin{itemize}
No relation; Left Neglect uses therapy for treatment
\end{itemize}
\item Depression and Multiple Personality Disorder
\begin{itemize}
\item Can have some effect because Left Neglect patients realize their disorder and are troubled within themselves
\end{itemize}
\item Bipolar disorder
\begin{itemize}
\item No relation because Left Neglect patients are still the same person
\end{itemize}
\item Autism and Asperger
\begin{itemize}
\item Some relation in the sense that both involve a disorder in the brain.
\item Differs in that Asperger's is a condition in which the language aspect of the brain (left) is affected, Left Neglect affects the right.
\end{itemize}
\end{itemize}
\section{Bipolar Disorder}
\begin{abstract}
Disease in which the patient has two diferent mental states: mania and depressive. Patient fluctuates in between the two. Diagnosis is difficult and the cause is not known, but could be related to inherited genetic traits, physiological abnormalities in the brain, chemical imbalance with neurological transmitters abd initiated by environmental stressors.
\end{abstract}
\begin{IEEEkeywords}
Mania, Depressive
\end{IEEEkeywords}
\subsection{Causes}
The direct cause of Bipolar Disorder is unknown, however, it can be related to inherited genetic traits, physiological differences in the brain, chemical imbalance with neurotransmitters and initiated by the environmental stressors.
\\
\vspace{2 mm}
Genetic causes:
\begin{itemize}
\item Familial: Half the people with bipolar disorder have a family member with it
\item A person with one parent with it has a 15-25 percent chance of developing it
\item A person with a fraternal twin with it has a 25 percent chance of developing, same risks as if both parents have it
\item A person with an identical twin with bipolar disorder has eightfold chance
\end{itemize}
\vspace{2 mm}
Neurochemcial Causes:
\begin{itemize}
\item Dysfunction with neurotransmitters and their respective chemicals
\item Norepinephrine
\item Serotonin
\item May lie dormant and can be activated on its own or triggered by external factors
\end{itemize}
Environmental Factors:
\begin{itemize}
\item Life event extremities
\item Altered health habits such as alcohol or drug abuse
\item Underdiagnosis in the past could explain the trend of BD at earlier ages
\item Substance abuse may not be a cause, but in can worsen the depressive state
\end{itemize}
\begin{IEEEkeywords}
Genetical/Familial, Neurochemical, Environmental
\end{IEEEkeywords}
\subsection{Symptoms}
Bipolar disorder is generally characterized by two episodes:
\\
\begin{itemize}
\item Manic: Extreme Happiness/Giddiness
\begin{itemize}
\item Hyperactive Mannerisms
\end{itemize}
\item Depressive: Extreme Sadness (Depression)
\begin{itemize}
\item Slower mannerisms
\end{itemize}
\end{itemize}
\vspace{2 mm}
The patient will fluctuate between these two episodes in violent mood swings. They can be in one mood for the whole day than switch to the other one spontaneously. It is dependent on the situation and the individual rather than anything concrete or genetically related.
\begin{IEEEkeywords}
Manic, Depressive
\end{IEEEkeywords}
\subsection{Treatments}
Treatment options include:
\begin{itemize}
\item Medications
\begin{itemize}
\item Mood Stabilizer (Lithium)
\item Antispychotics (Olanzapine)
\item Antidepressants (Fluoxetine)
\item Antidepressant-antipsychotics
\item Anti-anxieties
\end{itemize}
\item Psychotherapy: General term for treating mental health problems by talking with a psychiatrist
\item Electroconvulsive Therapy (ECT): Electrical currents passed through the brain to intentially cause a seizure. This wil alter the brain chemistry and thereby help revert certain mental illness symptoms.
\end{itemize}
\begin{IEEEkeywords}
Mood Stabilizers, Antipsychotics, Andtidepressants, Antidepressants-antipsychotics, anti-anxieties, psychotherapy, Electroconvulsive Therapy (ECT)
\end{IEEEkeywords}
\subsection{Special Characteristics}
Chracterized by drastic mood swings from periods of extreme highs to extreme lows. This complicates treatment options. The nature of the disease makes diagnosis difficult:
\begin{itemize}
\item Similar symptoms to other conditions
\item Difficulty with dealing with patients
\end{itemize}
\begin{IEEEkeywords}
Mood Swings
\end{IEEEkeywords}
\subsection{Relation to Other Diseases}
\begin{itemize}
\item Autism and Asperger
\begin{itemize}
\item Similar symptoms resulting in misdiagnosis
\end{itemize}
\item Depression and Multiple Personality Disorder
\begin{itemize}
\item Symptoms can coincide or be related
\end{itemize}
\item Effect of prescription and non prescription drugs an the brain
\begin{itemize}
\item Can cause or enhance similar symptoms
\end{itemize}
\item Left Neglect
\begin{itemize}
\item Both neurological, but very little similarity
\end{itemize}
\end{itemize}
\subsection{Pathology}
Has genetic linkage. Regions of interest include mutations on the chromosomes:
\begin{itemize}
\item 4p16
\item 12q23-q24
\item 16p13
\item 21q22
\item Xq24-q26
\end{itemize}
\begin{IEEEkeywords}
Chromosomal Mutation
\end{IEEEkeywords}
\subsection{Pathophysiology}
There is imbalance in neurotransmitters in the brain that lead to mood alterations. Furthermore, the prefrontal cortex (responsible for problem solving and decision making) in adults tends to be smaller and function less.
\begin{itemize}
\item Area matures during adolescence, which would explain appearance of disorder around a person's teen years
\end{itemize}
\begin{IEEEkeywords}
Mood Disorder, prefrontal cortex, neurotransmitters
\end{IEEEkeywords}

\section{Depression}
\begin{abstract}
A mood disorder that is defined by a continual feeling of sadness and a loss of interest in activities.
\begin{IEEEkeywords}
Sadness == symptoms, Depression == illness
\end{IEEEkeywords}
\subsection{Causes}
The major causes of depression include:
\begin{itemize}
\item Genetics
\item Neurochemistry
\item Environment
\end{itemize}
\subsubsection{Genetics}
Studies have indicated that people whose parents have depression have a higher chance of developing depression. This indicates that depression could be related to some familial trait or sex-linked trait. Likewise, studies have shown that a person whose twin has developed depression has a very high chance of developing it themselves. Finally, studies are working towards idenifying a gene responsible for depression.
\subsubsection{Neurochemistry}
Depression has been known to develop due to a chemical imabalance of neurotransmitters. Depression as an illness aroses due to the neurotransmitters:
\begin{itemize}
\item Norepinephrine
\item Serotonin
\item Dopamine
\end{itemize}
Each of these neurotransmitters are responsible for dealing with mood and other such mental states. Typically, either the receptor neuron or the nergic cell will become damaged, causing an underexpression of these neurotransmitters.
\begin{IEEEkeywords}
Genetics, Neurochemistry, Norepinephrine, Serotonin, Dopamine
\end{IEEEkeywords}
\subsection{Symptoms}
\begin{itemize}
\item Anxiety
\item Unhappiness
\item Loss of interest in activities
\item Loss of energy
\item Slowed thinking
\item Sleep Disturbance
\item Suicidal thoughts
\item Aches and pains
\item Weight problems
\item Increase risk of heart attack
\item Constricted blood vessels
\item Weakened immune system
\end{itemize}
\subsection{Types}
\begin{itemize}
\item Anxious distress
\item Mixed mood
\item Melancholic mood
\item Atypical mood
\item Psychotic mood
\item Catatonia
\begin{itemize}
\item Motionless for extended period of time
\end{itemize}
\begin{itemize}
\item Depresseion adjacent to birth
\end{itemize}
\item Seasonal pattern
\end{itemize}
\subsection{Treatment}
\begin{itemize}
\item Antidepressants drugs
\item Therapy
\end{itemize}
\subsection{Relationship to Other Diseases}
\begin{itemize}
\item Bipolar Disorder
\begin{itemize}
\item Neurochemical imbalance including norepinephrine, serotonin, and dopamine. Likewise, a person with bipolar disorder can have two extreme states: Mania (happiness), and Depressive -- Depression.
\end{itemize}
\item Left Neglect
\begin{itemize}
\item Damage to the brain in two areas, depression is a chemical imbalance not a physiological damage.
\end{itemize}
\item Prescription Drugs
\begin{itemize}
\item Prescription drugs such as antidepressants modify neurochemical levels by binding to the neuron receptors.
\end{itemize}
\end{itemize}
\section{Multiple Personality Disorder}
\begin{abstract}
A disordeer in which a person has two or more disting personalities that have power over their self.
\end{abstract}
\begin{IEEEkeywords}
Multiple personalities
\end{IEEEkeywords}
\subsection{Causes}
Multiple Personality Disorder is mainly a result of emotional, physical, and/or mental trauma through important events in childhood. For a person to be diagnosed with Multiple Personality Disorder, they must meet the following criteria:
\begin{itemize}
\item Two or more distinct identities or personality states are present, each with its own relatively enduring pattern of perceiving, relating to, and thinking about the environment and self.
\item Amnesia must occur, defined as gaps in the recall of everyday events, important personal information, and/or traumatic events.
\item The person must be distressed by the disorder or have trouble functioning in one or more major life areas because of the disorder.
\item The disturbance is not part of normal cultural or religious practices.
\item The symptoms can not be due to the direct physiological effects of a substance (such as blackouts or chaotic behavior during alcohol intoxication) or a general medical condition (such as complex partial seizures).
\end{itemize}
\begin{IEEEkeywords}
Result of severe emotional trauma
\end{IEEEkeywords}
\subsection{Symptoms}
\begin{itemize}
\item Lack of ability in explaining big parts of childhood events
\item Frequent short periods of memory loss
\item Sudden memory again
\item Feelings of being disconnected
\item Hallucinations
\item Suicide attempts
\item Varying levels of acting
\item Depression
\item Eating disorder
\item Headaches
\end{itemize}
\subsection{Treatment}
\begin{itemize}
\item Psychotherapy
\item Family or group therapy
\item Medications
\item Clinical hypnosis
\end{itemize}
\section{Prescription Drugs}
\begin{abstract}
Prescription drugs are durgs administerd by medical specialists to alleviate symptoms or aftereffects of a disease. Prescriptions can only be written by MD or DO, and are available through local pharmacy. These can often be abused, most often painkiller narcotics such as Vicodin, Oxycodone, and Diaudid. Nonprescription drugs are often abused and taken in large doses causing a lot of enyrolofical and physiological reactions.This type of drug abuse can be highly addictive and call for drug rehabilitation.
\end{abstract}
\begin{IEEEkeywords}
Prescription drugs, Vicodin, Oxycodone, Diaudid, Drug Rehabilitation
\end{IEEEkeywords}
\subsection{Non-prescription: Cocaine}
Cocaine is a stimulant made from the leaces of the coca plant. It affects dopamine within the vetnral tegmental area and nucleus accumbens, the want center. It is highly addictive. It works by binding to Dopamine transported proteins that uptake dopamine after synapse. This causes dopamine to accumulate in the synaptic cleft, increasing the activation of the dopamine receptors. The receptors are upregulated within the vetntral tegmental area and nucleus accumbens. This affects the sympathetic nervous system, increases heart rate and adrenaline production. The increased amount of dopamine causes cocaine users to feel great. However, this high only lasts a short amount of time, prompting dopamine receptor activity afterwards to be lessened causign users to redose to gain the affects of dopamine.
\subsubsection{Symptoms of Cocaine}
This causes symptoms of cravings and withdrawal that also characterizes depression. Cocaine can also cause tachycardia which can lead to arrhythmia and other heart conditions. This can also damage blood vessels in the kidneys. Cocaine can also damage nasal cavities and sinuses. Smoking crack can cause lung cancer. Constriction of blood vessels on the GI tract can cause ulcers and stomach and intestine damage. This can constrict blood vessels in the brain causing a stroke.
\begin{IEEEkeywords}
Coca plant, dopamine, ventral tegmental area, nucleus accumbens, dopamine synapse, synaptic cleft
\end{IEEEkeywords}
\subsection{Amphetamines}
\begin{itemize}
\item Central nervous stimulants prescribed for ADD and narcolepsy
\item Improve dopamine and norepinephrine neurotransmission
\item Used as recreational drugs or study aids
\item Derivatives: methamphetamine and MDMA.
\item Increase production of dopamine and norepniphreine by:
\begin{itemize}
\item Binding to synaptic terminal causing release of dopamine
\item Binding to synaptic vesicles in nerve terminals releasing stored dopamine
\item Binding to monoamine oxidase which prevents degradation of amphetamine within nerve terminals (Upregulation loop)
\item Binding to DAT proteings casuing dopamine to be pushed back into the synaptic cleft.
\item Increased release of noradrenaline and bind to noradrenaline reuptake rexeptors.
\end{itemize}
\item Act on sympathetic nervous system by increasing norepinephrine, increasing focus and concentration
\item Causes euphoria and happiness
\item Desired for increase in focus
\item Increates heart rate and blood pressure, dilating pupils
\item Reduces appetite, leading to weight loss with long term use
\end{itemize}
\subsection{Heroin}
\begin{itemize}
\item Powerful sedative made from opium poppy affecting dopamine levels in the ventral tegmental area and nucleus accumbens
\item Derivative of morphine, originally used for medical puposes
\item Passes through blood brain barrier readily and more potent than morphine
\item Converts to morphine and binds to the mu opiod receptor
\begin{itemize}
\item Decreases GABA prodution
\item Increases Dopamine production (GABA is a downregulator of dopamine)
\end{itemize}
\item Depressant that activates the parasympathetic nervous system:
\begin{itemize}
\item Decreased heart rate
\item Decreased breathing
\item Decreased mental function
\item Decrease symptoms of anxiety and depression
\end{itemize}
\item Very addictive for the above effects
\begin{IEEEkeywords}
Sedative from opium poppy, Morphine derivative, Blood brain barrier, Mu opiod receptor, Dopamine, ventral tegmental area, nucleus accumbens, GABA
\end{IEEEkeywords}
\subsection{Oxycodone}
\begin{itemize}
\item Opiod synthesized from thebaine
\item Used for analgesic purposes (painkilling)
\item Abused as a painkiller
\item Similar to Heroin
\item Binds to Kappa and Delta receptors instead
\item Also a depressant
\item Decreased GABA concentrations
\end{itemize}
\begin{IEEEkeywords}
Analgesic, Kappa and Delta receptors, GABA
\end{IEEEkeywords}
\subsection{Relationship to Other Diseases}
LOOK AT PRIOR DISEASES
\section{Alzheimer's Disease}
\begin{Abstract}
Alzheimer’s is a form of Dementia, a category of diseases defined by the following:
\begin{itemize}
\item Decline in memory and one of the following cognitive abilities severe enough to interfere with day-to-day living
\item Speech/language understanding and usage
\item Visual recognition (assuming no sensory damage)
\item Motor function
\item Problem Solving, abstract thinking (including decision making)
\item According to the Alzheimer’s Association (2013), Alzheimer’s accounts for 60-80 percent of all cases of dementia
\end{itemize}
\begin{IEEEkeywords}
Dementia, Decline in memory and cognitive abilities,
\end{IEEEkeywords}
\subsection{Characterizations}
\begin{itemize}
\item Progressive disease
\item On Average, death occurs 8 to 10 years after diagnosis
\item It is speculated that changes in the brain can occur up to 20 years before diagnosis
\item Characterized by protein deposits in the brain in the form of plaques and tangles
\item Disruption of cell communication causes cells in the brain to die
\end{itemize}
\begin{IEEEkeywords}
Protein deposits, plaques, tangles
\subsection{Symptoms at Diagnosis}
\begin{itemize}
\item Memory loss (Forgetting an acquaintance, etc)
\item Challenges in planning or problem solving
\item Confusion about the times or places
\item Spatial/visual confusion
\item New problems with language control/understanding
\item Frequently misplacing things
\item Poor judgment
\item Withdrawal from work or social activities
\item Mood/personality changes
\end{itemize}
\subsection{Late stage symptoms}
\begin{itemize}
\item Severe Memory Loss
\item Loss of Speech/Language understanding
\item Loss of mobility/motor skills
\item Severe weight loss
\item Incontinence
\item Visual or Auditory Hallucinations
\item Paranoia
\end{itemize}
\subsection{Causes}
It is not currently known what exactly causes the development of Alzheimer’s, but it is likely a combination of genetics, environmental, and lifestyle factors:
\begin{itemize}
\item Plaques:
\begin{itemize}
\item Beta-amyloid proteins generated by the breakdown of amyloid proteins found in the fatty membrane around nerve cells clump together in “plaques”.
\item These plaques block cell-signalling at synapses and may activate immune system cells,causing inflammation and destruction of the newly disabled cells.
\end{itemize}
\item Tau protein tangles:
\begin{itemize}
\item Tau is a protein that helps keep the brain’s nutrient transport system in orderly “tracks”.
\item When tau is twisted into tangles, it collapses, and the transportation “tracks” fall apart and disintegrate.
\item Due to this, Nutrients cannot reach cells and they die.
\end{itemize}
\end{itemize}
\begin{IEEEkeywords}
Plaques, Amyloid-Beta, Tau
\end{IEEEkeywords}
\subsection{Treatments}
There is no way to cure Alzheimer’s, but certain measures can be taken to lessen symptoms’ effects and improve quality of life.
\begin{itemize}
\item Cholinesterase inhibitors:
\begin{itemize}
\item Helps in the prevention of acetylocholine breakdown, which is a neurotransmitter that stimulates post-synaptic nerves, muscles and exocrine glands, playing a role in memory, motor function, and learning,, keeping its level relatively higher
\end{itemize}
\item Memantine:
\begin{itemize}
\item Regulates glutamine, which is released from the damaged brain cells and, when connected to NDMA, allows calcium into the brain, which can cause further damage
\end{itemize}
\item Antidepressants:
\begin{itemize}
\item Used to treat behavioral/emotional instability that often results in  Alzheimer's patients as well as sleep disorders
\end{itemize}
\item Assistive Care
\end{itemize}
\subsection{Effects on Body Systems}
\begin{itemize}
\item Infections of the respiratory system due to the inhalation of food caused by inability to swallow properly
\item Incontinence due to the loss of control of the bowels, leading to potential infections of the urinary tract
\item Loss of muscular control, which can lead patients to be bedridden and require help in simple everyday activities, such as bathing
\item Forgetfulness in areas such as grooming and hygiene open the doors for many infections to take hold of the body, which are the general cause of death in Alzheimer’s patients
\end{itemize}
\subsection{Relation to Other Diseases}
\begin{itemize}
\item Tay-Sachs Disease:
\begin{itemize}
\item Destruction of neurons and brain tissue, but only infants
\end{itemize}
\item Emotions and Memory:
\begin{itemize}
\item Affects emotional health
\begin{itemize}
\item Paranoia
\item Discomfort
\end{itemize}
\item Affects Memory
\begin{itemize}
\item Dementia
\end{itemize}
\end{itemize}
\item Parkinson’s Disease:
\item Another form of dementia, characterized by loss of motor control but without loss of brain tissue
\begin{itemize}
\end{itemize}
\item Huntington’s Disease:
\begin{itemize}
\item Loss of motor contol similar to PD and AD in respect to affect on brain, but is not characterized with dementia
\end{itemize}
\end{itemize}
\section{Parkinson's Disease}
\begin{abstract}
%\boldmath
Parkinson's Disease (PD) is a part of a group of disorders called motor system disorders. These disorders are classified as damage to neurons or part of the brain. In the case of PD, it results from loss of dopamine producers, dopaminergic cells in the striatal pathway. This is the basal ganglia part of the brain, specifically the substantia nigra pars compacta. It is mainly characterized by a resting tremor (shaking of body). There is no cure for PD, and treatment options are scarce, as well as the muscle degrades as time goes on.

\end{abstract}

\begin{IEEEkeywords}
Parkinson's Disease (PD), motor system disorders, dopaminergic nerves, resting tremor
\end{IEEEkeywords}

\subsection{Symptoms, Causes, and Treatments}
There is no known cause of PD. PD is marked by a loss of over 50 percent of the 200,000 dopaminergic neurons in the substantia nigra, usually due to damage of the basal ganglia.
There are three main symptoms of PD:
\begin{enumerate}
\item Resting Tremor
\item Akinesia (Rigidity of limbs)
\item Bradykinesia (Slowness of movement of limbs)
\end{enumerate}
Treatment options of PD are scarce. The main method of treating PD is to supplement the lost dopaminergic cells by one of two methods:
\begin{enumerate}
\item Dopamine agonists -- Compound that activates dopamine receptors in absence of their ligands.
\item Levodopa therapy -- Medicine that the brain converts into dopamine
\end{enumerate}
However, prolonged use of either treatment will casue the patient to oscillate in between functioning (that is being able to move and function normally) and inhibited (that is expressing the symptoms, rigidity, slowness and tremors of the muscle)
\begin{IEEEkeywords}
Dopaminergic nerves, substantia nigra, resting tremor, akinesia, bradykinesia, dopamine agonists, levodopa therapy
\end{IEEEkeywords}
\subsection{Pathology}
\subsubsection{Glial Cells}
A glial cell is primarily made of three parts: astrocytes, microglia, and oligodendrocytes. Oligodendrocytes are not involved in PD, but another related disease, multiple system atrophy. Astrocytes regulate synaptic activity and potassium extract concentrations. The function of microglia are relatively unknown. After injury, infection astrocytes and microglia are mobilized to respond to the infection. Astrocyte activation response is characterized by expression of glial fibrillary acidic protein (GFAP) near the injured area. Microglia are derived from macrophages outside the nervous system and do not affect other nervous system cells. Furthermore, activation of microglia is characterized by proliferation and expression of self-antigens and eventual transformation of microglia into phagocytes near the injured area. The precise function of microglia is still relatively unknown, they become mobilized and activated after injury or infection, and are used by the brain in response to cell death and injury. There has been a correlation between microglia activation and the neurodegenerative process that occurs in PD.
\subsubsection{Microglial Activation and Neurodegenerative Diseases}
The SN is a relatively rich area of microglia. Also, dopaminergic neurons have less of compound, glutathione, which serves a protection against oxidative stress and microglial-mediated injury. There are elevated levels of pro-inflammatory cytokines and oxidative stress-mediated damaged cells in PD afflicted patients, suggesting microglia activation. Occurrence of brain injury has also shown an increase in PD development risk, implying microglia activation as well. Furthermore, in patients with PD a large amount of microglia activation in areas with lots of dopaminergic neuron loss has been seen, suggesting some sort of correlation between the two. There has been a finding of upregulation of major histocompatibility complex (MHC) suggesting that inflammation is a vital component of PD. Also, there are increased numbers of HLA-DR, and MHC class I proteins. This suggests that because MHC is an antigen presenting protein, that microglia activation in PD afflicted patients is supported. Finally, there is an increase of pro-inflammatory ctyokines and glial derived cytokines such as EGF, TGF-α and bFGF which are essential to the niogrostriatal pathway (pathyway dopaminergic neurons follow). There is a correlation of between this increase and the pathogenesis of PD.
\begin{IEEEkeywords}
Glial, microglia, astrocytes, glial fibrillary acidic protein (GFAP), oxidative stress, pro-inflammatory cytokines
\end{IEEEkeywords}
\includegraphics[width=3.5in,height=13in,clip, keepaspectratio]{picture}
\begin{IEEEkeywords}
Figure of the pathology of microglial activation to death of the neuron. Lot's of complicated stuff here, but essentially what this says is that when microglial is activated, it causes the dopaminergic neuron to be oxidatively stressed and die
\end{IEEEkeywords}
\subsection{Physiology}
In the basal ganglia, the primary pathway that the dopamine produced by the SN leads to the striatum. The loss of dopamine to the striatum affects the pathway by:
\begin{enumerate}
\item Downregulation of the direct striatal pathway from GABAergic striatal neurons to the internal segment of the globus pallidus (GPi) and the substantia nigra pars retuclata (SNpr).
\item Upregulation of the indirect striatal pathway which instead uses the globus pallidus (GPe) and subthalamic nucleus (STN).
\end{enumerate}
Essentially, this causes less activity of the basal ganglia resulting in disrupt of the brain stem motor areas which is thought to be the primary cause of inhibited motor function (bradykinesia) in patients with PD, but does not accaunt for tremors and akinesia.
\begin{IEEEkeywords}
\textit{**Due to lots of names that make this really messy, this will include key concepts instead of terms:} Pathway for dopamine produced by the SN is altered causing disruption in the brain stem motor, thereby causimg bradykinesia.
\end{IEEEkeywords}
\includegraphics[width=3.5in,height=8in,clip,keepaspectratio]{physiology}
% use section* for acknowledgement
\vfill
\subsection{Connections to Diseases}
\begin{enumerate}
\item Alzheimer's (AD) and PD show lesions that kill neurons due to oxidative stress
\item PD and AD both develop with age as a key factor
\item Amyloid binding molecules and tau proteins have been seen in both PD and AD showing signs of dementia
\item PD and Huntington's are both neurodegenerative -- death of neurons, rigidity, etc.
\end{enumerate}
\includegraphics[width=3.5in,height=3in,clip,keepaspectratio]{prionlifnm_2165-f1}
\begin{IEEEkeywords}
This pathway shows the mutations that lead to PD, which are also commonly seen in AD. It's a bit more complicated than what's needed, but for the class's pleasure, it's included.
\subsection{PD Fact Chart}
\includegraphics[width=3.5in,height=9in,clip,keepaspectratio]{MJFF_WIP_MAR5_2}
\section{Huntington's Disease}

\end{document}